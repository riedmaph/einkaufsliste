\documentclass{article}

\begin{document}
\section{Paper prototyping}
We use paper prototyping to initially evaluate different options for displaying the list item:

\label{sec:list-types}
\begin{itemize}
    \item \textbf{Simple List:} List items are displayed beneath one another.\\
        \textbf{Variation 1:} The list is unsorted. We shuffled the inital list to prevent implicit sorting during list creation.
            The items are all the same size and same distance. \\
        \textbf{Variation 2:} The list is sorted implicitly such that items of the same category are grouped.
            The items are all the same size and same distance. \\
        \textbf{Variation 3:} The list is sorted such that items of the same category are grouped.
            The items are all the same size but there is greater space between the category-groups.
    \item \textbf{Map:} The app shows a zoomable, movable map of the market. Products to buy are placed as markers on the map. \\
        \textbf{Variation 1:} The markers are simply put on the map. No additional indicators are present.
        \textbf{Variation 2:} The markers are placed on the map. Markers that are outside the field of view are hinted using wedges. \cite{TODO}
\begin{itemize}

% TODO Images. We can just add the link to Drive or leave the images out of this document completely.

The evaluation took place in a simulated super market environment. 
Throughout the room there were scattered labeled stations/categories as found in a supermarket.
Each station had four to eight items usually found unter that category.

\subsection{Scenario}
The user was then given a shopping list of twelve or more items. 
He was instructed to ``buy'' the items on that list while describing aloud his actions, goals and thought process.

\subsection{Variables}
While varying the different list types as described above \ref{sec:list-types}, we observed the following parameters:
\begin{itemize}
    \item Number of times on the shopping list was looked at
    \item Number of items picked up between each look at the list
    \item Number of errors / corrections
    \item Distance walked
\end{itemize}

We also interviewed the participants after the shopping-runs.
The interviews were conducted semi-structured, using the following questions as guideline:
\begin{itemize}
    \item Which variant did you prefer and why?    
    \item Can you name advantages for each variant?
    \item How were the items partitioned / sorted? I.e. did they understand the implicit sorting.
    \item TODO
\end{itemize}

\subsection{Results}

% TODO

\section{Heuristic evaluation of Hi-Fi Prototypes}

In a subsequent iteration we rebuilt the prototypes in higher fidelity using the Proto.io platform.
The prototypes created were variations of the simple list described above. The map-variant was droped due to implementation complexity.
\begin{itemize}
    \item \textbf{Variation 1:} Roughly equivalent to variation one of the paper prototype iteration. 
        The list is unsorted, items are simply listed beneath one another.
    \item \textbf{Variation 2:} As with paper prototype variation 2, lists are sorted according to their categories. 
        They are still shown directly beneath on another with no hints as to the sorting.
    \item \textbf{Variation 3:} The high fidelity implementation of paper prototype variation 3 shows list items grouped by category.
        The grouping is explicitly given with titles for and a larger gap between the categories.
    \item \textbf{Variation 4:} A fourth variation was added that too shows the list items grouped into categories with title.
        This variant however puts the categories on different pages. Swiping left and right on a list allows switching between list.
        A page indicator shows that there are more pages available and which page is currently shown in the viewport.
\end{itemize}

Using the ten Usability Heuristics by Nielsen, we evaluated these improved iterations.

    \subsection{Results}

\section{User study with Hi-Fi Prototypes}

The results of the Heuristic Evaluation we then incorporated into the original Hi Fi Prototypes in another iteration.
These prototypes are the basis for a user study to assess their usability and general support for grocery shopping.

\subsection{Evaluation}
    \subsubsection{Hypotheses}
        Varying only prototype used:
        \begin{itemize}
            \item Random requires the highest number of interactions
            \item Random requires the longest distance walked 
            \item Random requires the longest time
            \item Random produces the highest error rate

            \item Grouped requires the lowest number of interactions

            \item Grouped+Swipe requires the shortest distance walked 
            \item Grouped+Swipe requires the shortest time
            \item Grouped+Swipe produces the lowest error rate
        \end{itemize}
        \begin{itemize}

            \item With Undo: Fewer errors / forgotten items
            \item Without hiding: Fewer errors / forgotten items
            \item With hiding: Faster

            \item Very small categories or very large categories: Slow and more errors

            \item Many interactions correlate with long time
            \item Long distance correlates with long time

            \item Errors impairs the fun

        \end{itemize}
    \subsubsection{Variables}
        \subsubsubsection{Independent variables}
            \begin{itemize}
                \item Prototype used
                    \item With or without Undo
                    \item Hide checked items or not
                \item Granularity of categories / number of categories 
                \item Total list length
            \end{itemize}
        \subsubsubsection{Dependent variables}
            \begin{itemize}
                \item Fun / Preference (Questionaire) % TODO find questionare
                \item Ease of Use (SUS + awesome Questionaire)
                \item Interaction frequency (how often does a user interact with the list) (counted by examiners || device (eyetracking?))
                \item Error rate (counted by examiners)
                \item Total time / Time saved (measured by examiners)
                \item Total distance walked (measured by step counter)
            \end{itemize}

    \subsubsection{Scenario/Setting + Setup}   
        siehe Bauer-Raum, nur halt echter Supermarkt, ne

        Within subject, everyone tries every prototyp in different Order.

        Latin Squared
    \subsubsection{Challenges / External factors / T2Valid}
        \begin{itemize}
            \item Order of categories
            \item Users usual shopping behaviour (items the user actually buys a lot might be easier to remember or vice versa)
            \item 
        \end{itemize}

    \subsubsection{Questionare}
        \textbf{Part 1: Usability (SUS)} % https://www.usability.gov/how-to-and-tools/methods/system-usability-scale.html
        \begin{itemize}
            \item I think that I would like to use this system frequently.
            \item I found the system unnecessarily complex.
            \item I thought the system was easy to use.
            \item I think that I would need the support of a technical person to be able to use this system.
            \item I found the various functions in this system were well integrated.
            \item I thought there was too much inconsistency in this system.
            \item I would imagine that most people would learn to use this system very quickly.
            \item I found the system very cumbersome to use.
            \item I felt very confident using the system.
            \item I needed to learn a lot of things before I could get going with this system.
        \end{itemize}
        \textbf{Part 1b: USE Questionaire} % garyperlman.com/quest/quest.cgi?form=USE

        \textbf{Part 2: Fun} % http://dare.ubvu.vu.nl/handle/1871/9782
        \begin{itemize}
            \item Do you work with the program without someone telling you to?
            \item Would you like to work with the program when other children can decide for themselves what to do?
            \item Do you think it is boring to work with the program?
            \item When you started working with the program, did you want to continue working with it?
            \item Do you think your friends would like the program?
            \item Do you think the program is childish?
            \item Is the program is too difficult to play with?
            \item When you have worked with the program once, does it remain fun?
            \item Do you enjoy yourself when you are working with the program?
            \item Does the program contain many surprises?
            \item Would you like to work with the program more often?
            \item Do you perform well on the exercises in the program?
            \item Would you like to have the program at home?
            \item Do you make many mistakes while you are working with the program?
        \end{itemize}
        \textbf{Part 3: Tech Affinity & Shopping behaviour}

        \textbf{Part 4: Demographics}
        \begin{itemize}
            \item Age
            \item Gender
            \item Marital status
            \item Household size
            \item (Household-)Income

        \end{itemize}
% TODO

\end{document}

