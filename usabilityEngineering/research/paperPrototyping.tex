\documentclass{article}

\begin{document}
\section{Paper prototyping}
We use paper prototyping to initially evaluate different options for displaying the list item:

\label{sec:list-types}
\begin{itemize}
    \item \textbf{Simple List:} List items are displayed beneath one another.\\
        \textbf{Variation 1:} The list is unsorted. We shuffled the inital list to prevent implicit sorting during list creation.
            The items are all the same size and same distance. \\
        \textbf{Variation 2:} The list is sorted implicitly such that items of the same category are grouped.
            The items are all the same size and same distance. \\
        \textbf{Variation 3:} The list is sorted such that items of the same category are grouped.
            The items are all the same size but there is greater space between the category-groups.
    \item \textbf{Map:} The app shows a zoomable, movable map of the market. Products to buy are placed as markers on the map. \\
        \textbf{Variation 1:} The markers are simply put on the map. No additional indicators are present.
        \textbf{Variation 2:} The markers are placed on the map. Markers that are outside the field of view are hinted using wedges. \cite{TODO}
\begin{itemize}

% TODO Images. We can just add the link to Drive or leave the images out of this document completely.

The evaluation took place in a simulated super market environment. 
Throughout the room there were scattered labeled stations/categories as found in a supermarket.
Each station had four to eight items usually found unter that category.

\subsection{Scenario}
The user was then given a shopping list of twelve or more items. 
He was instructed to ''buy'' the items on that list while describing aloud his actions, goals and thought process.

\subsection{Variables}
While varying the different list types as described above \ref{sec:list-types}, we observed the following parameters:
\begin{itemize}
    \item Number of times on the shopping list looked at
    \item Number of items picked up between each look at the list
    \item Number of errors / corrections
    \item Distance walked
\end{itemize}

We also interviewed the participants after the shopping-runs.
The interviews were conducted semi-structured, using the following questions as guideline:
\begin{itemize}
    \item Which variant did you prefer and why?    
    \item Can you name advantages for each variant?
    \item How were the items partitioned / sorted? I.e. did they understand the implicit sorting.
    \item TODO
\end{itemize}
 
\subsection{Results}
 Results will be discussed after further study. This document is currently in an intermediate state and will be adapted further on.
\end{document}
